\section{提交订单}

提交订单的逻辑整体上复用 Ukey 新购逻辑。

\subsection{后台配置}

\paragraph*{申请主体}

控制可进行 Ukey 变更的主体,包括内外部单位,内外部个人。如果前台登陆的用户不满足主体筛选条件,则不展示 Ukey 变更入口。前台 Ukey 变更入口(侧边栏菜单)由后台接口控制。

\paragraph*{通道选择}

标准包仅支持契约锁通道,该配置项主要用于项目定制化需求。

\paragraph*{支付方式}

Ukey 变更配置项相比 Ukey 新购少一个烧制方式,配置完支付方式后自动选择对应的烧制方式。主要支付方式的影响如表\ref{table:支付方式配置项}:

\begin{table}[H]
  \small
  \centering
  \caption{支付方式配置项}
  \label{table:支付方式配置项}
  \setlength{\tabcolsep}{4mm}
  \begin{tabular}{c|cc}
    \toprule
    \textbf{支付方式} & \textbf{含义} & \textbf{烧制方式} \\
    \midrule
    支付给平台方 & 私有化费用分摊,走私有云烧制流程 & 平台方烧制/用户自主烧制 \\
    支付给契约锁 & 契约锁收费,走公有云烧制流程 & 契约锁客服烧制 \\
    \bottomrule
  \end{tabular}
\end{table}

此外,还可能出现\textbf{不支付}选项,这代表私有云系统启用了费用分摊功能,但没有配置相关的规则,此时仍然走私有云烧制流程。

\paragraph*{可用证书}

可用证书包括可用 RSA 证书与可用 SM2 证书,该配置项会出现三种情形:
\begin{itemize}
    \item 无可用 RSA/SM2 证书: 自建通道禁止某一项,则只给我文本提示: 当前自建通道无法申请当前算法长期证书
    \item 有特定的 RSA/SM2 证书范围: 通过多选组件选择可用的 CA 机构
    \item 能开启 RSA/SM2 证书: 通过开关控制能否启用 RSA/SM2 证书
\end{itemize}

\paragraph*{可选年限}

可选年限包括一年期与两年期,可以选择一个或所有。如果都不选,那么前台不会展示 Ukey 变更入口。

\paragraph*{是否需要烧制章面}

该配置项有三个单选项: 需要章面,无需章面,由用户自主选择。前两个选项会强制用户制作/不制作章面。

上述的所有配置会在用户提交订单时生成一份快照,即以用户提交时的配置为准。如果后续更改了配置,不影响先前已提交的订单。

\subsection{Ukey 变更入口}

Ukey 变更前台界面的路由为: \texttt{/cert-apply?orderType=UKEY\_CHANGE}。所有 Ukey 变更入口都通过改变路由进入变更界面。目前主要有以下几类入口:
\begin{itemize}
    \item 常规申请: 前台证书界面,点击证书服务的 Ukey 变更按钮
    \item 跳转申请: 首页,个人中心,合同签署等界面跳转至 Ukey 变更界面
    \item 开放平台: 开放平台生成路由链接跳转至 Ukey 变更界面
\end{itemize}

常规申请是最常见的 Ukey 变更订单申请方式,进入变更界面后,系统将获取后台配置并提供各字段的选项。

跳转申请会根据来源不同对部分字段作限制,需要限制的字段会出现在路由上,例如通过 \texttt{userCenterSigAlgType} 选中默认算法,通过 \texttt{renew} 知道来自个人中心,需要进行相关的初始化操作。

开放平台申请的链接在路由上会有 \texttt{viewToken} 字段,进入界面后,系统会根据 \texttt{viewToken} 查询相关配置,并对订单信息做一定限制。

三类入口填写证书信息获取配置与进行限制的逻辑如图\ref{fig:提交订单入口配置逻辑}:
\begin{figure}[H]
  \small
  \centering
  \begin{tikzpicture}[font=\small, align=center]
    \node(common) at (0,1) {常规申请};
    \node(jump) at (0,0) {跳转申请};
    \node(app) at (0,-1) {开放平台};
    \node(commit) at (12,1) {填写信息};
    \node(oss) at (9,1) {获取信息配置};
    \node(url) at (6,0) {路由参数限制};
    \node(viewToken) at (3,-1) {viewToken限制};
    \draw [-Stealth] (common) -- (oss);
    \draw [-Stealth] (jump) -- (url);
    \draw [-Stealth] (url) -| (oss);
    \draw [-Stealth] (app) -- (viewToken);
    \draw [-Stealth] (viewToken) -| (url);
    \draw [-Stealth] (url) -| (oss);
    \draw [-Stealth] (oss) -- (commit);
  \end{tikzpicture}
  \caption{提交订单入口配置逻辑}
  \label{fig:提交订单入口配置逻辑}
\end{figure}

\subsection{Ukey 信息填写}

\subsubsection{指定需要变更的 Ukey}

目前支持两种指定 Ukey 的方式:选择Ukey与插入Ukey。H5 仅支持选择 Ukey。

\paragraph*{选择Ukey} 

系统将返回\textbf{当前私有化系统}中用户身份所管理的\textbf{订单的Ukey}列表,如果Ukey存在变更中的订单,则置灰不允许选中,否则可以选中。选中后点击详情可获得Ukey相关的信息。

\paragraph*{插入Ukey} 用户插入一个契约锁颁发的 Ukey,系统将自动识别并获取其相关信息。在插入 Ukey 模式下,即使是\textbf{非当前私有化系统}的 Ukey 也可进行变更。非当前私有化系统的 Ukey 无法通过接口获取完整的信息,因此 Ukey 信息会出现显示不全的情形。

\subsubsection{其他信息填写}

其他信息与 Ukey 新购基本一致,Ukey 变更包含两部分信息: 新证书信息与章面信息(如果禁止制章则不存在章面信息)。

用户在选择单位/个人时,列表会给出认证状态,如果主体未认证,则需完成认证流程。如果选中的主体与当前使用的身份不符,则会通知对方授权,授权后继续支付流程。

用户在填写完基本信息后,界面右上角会显示 Ukey 变更的价格,点击提交订单即可。

如果变更的 Ukey 在当前私有化系统中存在章面信息,提交订单后会给出相关提示。

\subsection*{详细逻辑图}

提交 Ukey 变更订单的详细逻辑如图\ref{fig:提交订单详细逻辑}所示:

\begin{figure}[H]
    \small
    \centering
    \begin{tikzpicture}[font=\small, align=center]
      \node [color=gray, font=\footnotesize] (oss1) at (1.5,1.25) {后台控制};
      \node (app) at (0,0) {前台申请};
      \node (limit) at (3,0) {入口\\\footnotesize{(限制逻辑)}};
      \node [font=\footnotesize] (jump) at (2,-1.5) {跳转申请};
      \node [font=\footnotesize] (open) at (4,-1.5) {开放平台};
      \node [font=\footnotesize] (select) at (6.5,0.75) {选择Ukey\\\scriptsize{(仅当前系统Ukey)}};
      \node [font=\footnotesize] (insert) at (6.5,-0.75) {插入Ukey\\\scriptsize{(任意契约锁Ukey)}};
      \node [color=gray, font=\footnotesize] (oss2) at (10,1.25) {后台配置};
      \node (info) at (10,0) {填写信息};
      \node (commit) at (13,0) {提交订单};
      \draw [-Stealth] (app) -- (limit) node [midway, below, font=\scriptsize] {是否显示};
      \draw [dashed, color=gray] (oss1) -- (1.5, 0);
      \draw (jump) -- (limit);
      \draw (open) -- (limit);
      \draw [-Stealth] (limit) -- ++(1.5,0) -- (select);
      \draw [-Stealth] (limit) -- ++(1.5,0) -- (insert);
      \draw [-Stealth] (select) -- ++(1.75,-0.75) -- (info);
      \draw [-Stealth] (insert) -- ++(1.75,0.75) -- (info);
      \draw [dashed, color=gray] (oss2) -- (info);
      \draw [-Stealth] (info) -- (commit) node [midway, below, font=\scriptsize] {确认变更};
    \end{tikzpicture}
    \caption{提交订单详细逻辑}
    \label{fig:提交订单详细逻辑}
\end{figure}

\newpage